\documentclass[10pt]{article}

% Packages.
\usepackage{amsfonts,amsmath,amssymb}
\usepackage{fancyhdr}
\usepackage[margin=1in]{geometry}
\usepackage{xcolor}

% Style the headers and footers.
\fancypagestyle{firststyle}
{
   \fancyhf{}
   \rfoot{\thepage}
}
\fancypagestyle{nonfirststyle}
{
   \fancyhf{}
   \lhead{Elegant Derivatives of Large Products}
   \rhead{Last Updated: November 30, 2021}
   \cfoot{\thepage}
}
\pagestyle{nonfirststyle}

\title{Elegant Derivatives of Large Products}
\author{Joshua J. Daymude}
\date{November 30, 2021}

\begin{document}

\thispagestyle{firststyle}

\maketitle




\section*{Introduction}

This short writeup details two derivations of the solution:
\[\frac{d}{dx}\prod_{i=1}^n f_i(x) = \prod_{i=1}^n f_i(x) \cdot \sum_{i=1}^n \frac{f_i'(x)}{f_i(x)}\]
These derivations also work for infinite products:
\[\frac{d}{dx}\prod_{i=1}^\infty f_i(x) = \prod_{i=1}^\infty f_i(x) \cdot \sum_{i=1}^\infty \frac{f_i'(x)}{f_i(x)}\]




\section*{Method 1: Product Rule}

The product rule states:
\[\frac{d}{dx}(u \cdot v) = u' \cdot v + u \cdot v'\]
By iteratively peeling off terms from the product and applying the product rule, we obtain:
\begin{align*}
    \frac{d}{dx} \prod_{i=1}^n f_i(x) &= \frac{d}{dx} \left(f_1(x) \cdot \prod_{i=2}^n f_i(x)\right) \\
    &= f_1'(x) \cdot \prod_{i=2}^n f_i(x) + f_1(x) \cdot \frac{d}{dx} \left(\prod_{i=2}^n f_i(x)\right) \\
    &= f_1'(x) \cdot \prod_{i=2}^n f_i(x) + f_1(x) \cdot \left(f_2'(x) \cdot \prod_{i=3}^n f_i(x) + f_2(x) \cdot \frac{d}{dx} \left(\prod_{i=3}^n f_i(x)\right)\right) \\
    &= f_1'(x) \cdot \prod_{i=2}^n f_i(x) + f_1(x) \cdot \left(f_2'(x) \cdot \prod_{i=3}^n f_i(x) + f_2(x) \cdots \right. \\
    &\quad \left. + f_{n-2}(x) \cdot \left(f_{n-1}'(x) \cdot f_n(x) + f_n'(x) \cdot f_{n-1}(x)\right) \cdots \right)
\end{align*}
Distributing the singular $f_i(x)$ terms into their following nested sums and rearranging, we obtain
\[\frac{d}{dx} \prod_{i=1}^n f_i(x) = \sum_{i=1}^n f_i'(x) \cdot \frac{\prod_{j=1}^n f_j(x)}{f_i(x)}
= \prod_{i=1}^n f_i(x) \cdot \sum_{i=1}^n \frac{f_i'(x)}{f_i(x)}.\]




\section*{Method 2: Leveraging Logarithms}

Let $F(x) = \prod_{i=1}^n f_i(x)$.
Then taking the natural logarithm of both sides yields
\[\ln(F(x)) = \ln\left(\prod_{i=1}^n f_i(x)\right) = \sum_{i=1}^n \ln(f_i(x))\]
The derivative of the natural logarithm is $\frac{d}{dx}\ln x = 1/x$.
So, taking the derivative of both sides and observing that the derivative of a finite sum is equal to the finite sum of derivatives,
\begin{align*}
    \frac{d}{dx}\ln(F(x)) &= \frac{d}{dx}\sum_{i=1}^n \ln(f_i(x)) \\
    \frac{1}{F(x)} \cdot \frac{dF}{dx} &= \sum_{i=1}^n \frac{1}{f_i(x)} \cdot f_i'(x) \\
    \frac{dF}{dx} &= F(x) \cdot \sum_{i=1}^n \frac{f_i'(x)}{f_i(x)}
\end{align*}
Substituting the full expression back in for $F(x)$ yields
\[\frac{d}{dx}\prod_{i=1}^n f_i(x) = \prod_{i=1}^n f_i(x) \cdot \sum_{i=1}^n \frac{f_i'(x)}{f_i(x)}\]




\section*{Notes and Caveats}

\begin{itemize}
    \item The second method only applies when $f_i(x) > 0$ for all $i$ and $x$; otherwise, $\ln(f_i(x))$ is undefined.
    The first method holds in all cases.

    \item The second method relies on the derivative of a sum being equal to the sum of the derivatives.
    When applied to an infinite sum, this is true if (a) $f_i$ is differentiable over the domain for all $i$, (b) $\sum_{i=1}^\infty f_i'(x)$ uniformly converges, and (c) there exists at least one point of convergence for $\sum_{i=1}^\infty f_i(x)$.
    Unless I'm mistaken, the first method does not have this problem and can be directly applied to infinite products.
\end{itemize}


\end{document}
